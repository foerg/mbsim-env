\section{Introduction}
\MBSim{} is a simulation tool to analyse the dynamic phenomenons of dynamical systems. Its root is the modeling of nonsmooth multibody systems. The mathematical background has been developed at the Institute of Applied Mechanics of the Technische Universit\"at M\"unchen. \MBSim{} is developed further at the institute and by a private interest group. A summary concerning rigid body dynamics is given in the PhD thesis of Martin F\"org~\cite{Foer07} and the lecture notes~\cite{Sch14c}. The PhD thesis of Roland Zander~\cite{Zan09} introduces the theory of flexible bodies. Ref.~\cite{Zan08} shows an overview about the research at the institute. This reference includes simulation results of academic and industrial examples~\cite{Ceb14,Gru15,Sch10}. Extensions regarding hydraulics~\cite{Sch12g} and signal processing as well as parallelisation and cosimulation~\cite{Fri11,Cla13} are unique in the field of nonsmooth dynamical systems.
%
\subsection{Features}
The \MBSim{} environment consists of four programs, \MBSim{}, \OpenMBV{}, \HDFSerie{} and \FMatVec{}.\par
%
\FMatVec{} is a template-based matrix-vector library. Depending on the size of the matrices, calculations are either undertaken directly or provided by Lapack and Blas. One can also link to (parallel) Atlas or (parallel) Intel MKL. Reference counting is done either with boost::shared\_array or with a builtin process.\par
%
\HDFSerie{} is a time series wrapper for \HDF. It offers also a plotting tool for graphical visualisations.\par
%
\OpenMBV{} is a visualization tool for multibody simulation. It can handle large files with \HDF, Octave preprocessing, easy-to-handle XML, analytic curve representations and various interfaces with swig. The visualization is based on Coin.\par
%
\MBSim{} is a tool for the simulation of dynamical systems. The kernel offers the simulation of multibody systems with impacts and friction. Surfaces may be described by Nurbs-interpolation. Different modules exist for control, hydraulics, flexible multibody systems with small and large deflections, electronics, powertrain applications and client-server simulations. Models can be written with C++ but also with XML. XML is also the basis for a GUI, which offers easy drag-and-drop modeling. An interface for FMI model export is available.
%
\subsection{Objective}
The goal of this introduction is to motivate the use of \MBSim{}. It shows the installation of the necessary program parts, describes basically its components and program flow and gives some examples. You find it in the Download section of the \MBSim{} webpage, where you can also find binary releases for Linux and Windows. For more information, e.g. Doxygen description and current build status, visit the {\href{https://www.mbsim-env.de/mbsim/}{\textsf{Official Build System of the MBSim-Environment}}}.
%
\subsection{Acknowledgement}
Many people have been helping to implement \MBSim{}. Among them have been students and PhD students of the Institute of Applied Mechanics of the Techische Universit\"at M\"unchen. We mention the former PhD students, in alphabetical order, Mathias Bachmayer, Thomas Cebulla, Jan Clauberg, Bastian Esefeld, Martin F\"org, Markus Friedrich, Kilian Grundl, Robert Huber, Thorsten Schindler, Markus Schneider, Roland Zander. 

