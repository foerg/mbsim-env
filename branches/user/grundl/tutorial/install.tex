\section{Installation}
This section summarizes the necessary steps to install \MBSim{}.

%%------------------------------------------------------------ SUBSECTION ---------------------
\subsection{Where To Find the Source Code}
The source code of \MBSim{} together with some examples, the necessary \FMatVec{} library, a \HDF{} wrapper for output and the visualisation program \OpenMBV{} can be found at \url{http://code.google.com} using subversion code administration\footnote{SVN Quick Reference Card: \url{http://www.cs.put.poznan.pl/csobaniec/edu/svn-refcard.pdf}}. Further, one needs \HDF{} from \url{http://www.hdfgroup.org}. Everything is placed under \href{http://www.gnu.org/licenses/lgpl.html}{LGPL}\footnote{see file~\texttt{COPYING} in the root directory of the specific source code}.\par

%%------------------------------------------------------------ SUBSECTION ---------------------
\subsection{Installation Procedures}
For the installation of the specific projects always the same \emph{procedures} have to be applied. They are summarised in the following.

\subsubsection{Installation}
\begin{itemize}
	\item \textsc{automake}:
	\begin{itemize}
		\item[] \begin{verbatim}autoreconf -fi\end{verbatim}
	\end{itemize}
	\item \textsc{configure}: 
	\begin{itemize}
		\item[] \begin{verbatim}./configure \end{verbatim}
        \item[] with defining a location for the installation 
                \begin{verbatim}--prefix=$HOME/.../Install\end{verbatim}
        \item[] possibly with FLAGS for debug information
		        \begin{verbatim} CFLAGS="-g3 -O0" CXXFLAGS="-g3 -O0" F77FLAGS="-g3 -O0" FFLAGS="-g3 -O0" \end{verbatim}
		\item[] possibly with project depending FLAGS
	\end{itemize}
	\item \textsc{install}
	\begin{itemize}	
		\item \begin{verbatim}make\end{verbatim}
		\item \begin{verbatim}make install\end{verbatim}
	\end{itemize}
\end{itemize}
All procedures belong to the GNU-Build-System (cf.~Sec.~\ref{sec:gnu}).\par
It is reasonable to write an executable script file invoking the procedures.

\subsubsection{Reinstallation/Update}
The procedure \textsc{reinstall}
\begin{verbatim}
 make uninstall 
 make clean
 svn update // only for Update
 ./config.status --recheck
 make install
\end{verbatim}
newly installs a project with the same configure options used at the previous installation. These informations are stored in the file config.status.\par 

The procedure \textsc{update} is similar to the reinstallation with the additional 
\begin{verbatim} 
 svn update 
\end{verbatim}
before the recheck option.\par

For restoring a not-configured version of the project
\begin{verbatim}
 make maintainer-clean
\end{verbatim}
is used. After that \textsc{configure} has to be invoked again.

\subsubsection{Uninstallation}
For uninstalling
\begin{verbatim}
 make uninstall
 make clean
\end{verbatim}
has to be called in all directories. If the files created by configure should be deleted too, type 
\begin{verbatim} 
 make distclean 
\end{verbatim} 
in addition.

%%------------------------------------------------------------ SUBSECTION ---------------------
\subsection{Installation of the Simulation Framework\label{sec:install:simulation}}
In the following it is assumed, that a directory~\texttt{MBSim} and a directory \texttt{MBSim/Install} has been created in the \texttt{\$HOME} path of the Linux operating system.\par
All projects depend on PKG package administration. That is why the file \texttt{\$HOME/.bashrc} has to be extended with
\begin{verbatim}
 export PKG_CONFIG_PATH=
        "$HOME/MBSim/Install/lib/pkgconfig/:$PKG_CONFIG_PATH"
 export LD_LIBRARY_PATH=
        "$HOME/MBSim/Install/lib/:$LD_LIBRARY_PATH"
\end{verbatim}

\subsubsection{\FMatVec{}}
\FMatVec{} is a library for fast matrix-vector evaluations based on LAPack/Blas, ATLAS or the IntelMKL in its sequential or parallel version.\\
For the installation the following instructions have to be completed:
\begin{verbatim}
 cd $HOME/MBSim
 svn checkout http://fmatvec.googlecode.com/svn/trunk/ fmatvec
 cd $HOME/MBSim/fmatvec
\end{verbatim}
Continue with the procedure \textsc{automake}.\par
Then, the procedure \textsc{configure} for dynamic compilation is used with the prefix:
\begin{verbatim}
--prefix=$HOME/MBSim/Install
\end{verbatim}

By default, Lapack/Blas is used. If Lapack/Blas is not located in a standard search path, the following FLAGS must be used:
\begin{verbatim}
 --with-blas-lib-prefix=PFX (prefix, where the BLAS lib is
     installed, when ATLAS is not used)
 --with-lapack-lib-prefix=PFX (prefix, where the LAPACK lib is
     installed, when ATLAS is not used)
\end{verbatim}

If ATLAS should be used instead of Lapack/Blas, the following FLAGS must be used:
\begin{verbatim}
 --enable-atlas (use ATLAS)
 --with-atlas-inc-prefix=PFX (prefix, where the ATLAS includes 
     are installed, when ATLAS is used)
 --with-atlas-lib-prefix=PFX (prefix, where the ATLAS libs are
     installed, when ATLAS is used)
\end{verbatim}

If the IntelMKL should be used instead of Lapack/Blas, the following FLAGS must be used:
\begin{verbatim}
 --enable-intelmkl (use IntelMKL)
 --with-intelmkl-inc-prefix=PFX  (prefix where IntelMKL includes
     are installed, when the IntelMKL is used)
 --with-intelmkl-lib-prefix=PFX  (prefix where IntelMKL libraries are
     installed, when the IntelMKL is used)
\end{verbatim}

By default the sequential version of the IntelMKL is used. For using the parallel version of the IntelMKL, set the following FLAG:
\begin{verbatim}
  --enable-parallelIntelMKL
\end{verbatim}

To deactivate the OpenMP pragma \textit{omp critical} within reference counting use
\begin{verbatim}
--disable-pragma_omp_critical
\end{verbatim}
The code can be compiled and installed with a Doxygen HTML class documentation by \texttt{make doc} and the procedure \textsc{install}.

\subsubsection{\HDF}
\HDF{} is a hierarchical data format enabling the effective administration of plot and visualisation data. It can be downloaded as \textbf{source code} ("ALL Platforms") 
from \url{http://www.hdfgroup.org/HDF5/} with at least version 1.8.2.\par
Extract the source archive to \texttt{\$HOME/MBSim/hdf5}.\par
Change to \texttt{\$HOME/MBSim/hdf5}.\par
Use the procedure \textsc{configure} for dynamic compilation with the prefix
\begin{verbatim}
--prefix=$HOME/MBSim/Install
\end{verbatim}
and the additional FLAG
\begin{verbatim}
 --enable-cxx
\end{verbatim}
Compilation is done with the procedure \textsc{install}.\par
A \HDF{} wrapper makes it possible to use \HDF{} very easily. It is available by
\begin{verbatim}
 cd $HOME/MBSim
 svn checkout http://hdf5serie.googlecode.com/svn/trunk/ hdf5serie
\end{verbatim}
For having \MBSim{} creating \HDF{} files invoke
\begin{verbatim}
 cd $HOME/MBSim/HDF5Serie/hdf5serie
\end{verbatim}
as well as the procedures \textsc{automake, configure} for dynamic compilation with the prefix
\begin{verbatim}
--prefix=$HOME/MBSim/Install
\end{verbatim}
\texttt{make doc} and \textsc{install} for installation and creation of a Doxygen HTML class documentation.\par
Last, \texttt{.bashrc} can be extended with
\begin{verbatim}
alias h5lsserie="$HOME/MBSim/Install/bin/h5lsserie"
alias h5dumpserie="$HOME/MBSim/Install/bin/h5dumpserie"
\end{verbatim}
to gain overall access to the commands \texttt{h5lsserie} and \texttt{h5dumpserie}.

\subsubsection{\OpenMBV{}}
\OpenMBV{} visualises \MBSim{} simulations using XML and \HDF{} in a coinciding hierarchical structure. The installation for the simulation framework consists of two steps: first the XML utils have to be installed, then \MBSim{} needs \textsf{OpenMBV-C++Interface} to create standard data for \OpenMBV{} using C++ programs. The source code is available by
\begin{verbatim}
 cd $HOME/MBSim
 svn checkout http://openmbv.googlecode.com/svn/trunk/ openmbv
\end{verbatim}

\paragraph{XML Utils}
It is assumed that Octave with version 3.0 or newer, Xerces and boost are installed.\par
Then,
\begin{verbatim}
 cd $HOME/MBSim/OpenMBV/mbxmlutils
\end{verbatim} 
and use the procedures \textsc{automake}, \textsc{configure} for dynamic compilation with the prefix
\begin{verbatim}
--prefix=$HOME/MBSim/Install
\end{verbatim}
and (if not using Casadi) FLAG \texttt{--enable-only-mbxmlutilstinyxml} and \textsc{install} for installation of an independent XML preprocessor to parse and validate hierarchical XML-files.

\paragraph{OpenMBV-C++Interface}
It is assumed that 
\begin{itemize}
\item hdf5 with version 1.8.2 or newer
\item HDF5Serie 
\end{itemize}
is installed.\par
Invoke
\begin{verbatim}
 cd $HOME/MBSim/OpenMBV/openmbvcppinterface
\end{verbatim} 
and the procedures \textsc{automake, configure} for dynamic compilation with the prefix
\begin{verbatim}
--prefix=$HOME/MBSim/Install
\end{verbatim}
Sometimes trouble with linking \emph{swig} occurs; in this case just set
\begin{verbatim}
--with-swigpath
\end{verbatim}
to some value such that \emph{swig} is not found on your system.
\texttt{make doc} and \textsc{install} for installation and creation of a Doxygen HTML class documentation. 

\subsubsection{\MBSim}
Necessary for the installation of \MBSim{} are
\begin{itemize}
\item \FMatVec{}
\item \OpenMBV{}-C++-Interface
\end{itemize}
and Casadi from \texttt{https://github.com/casadi/casadi/wiki} until \texttt{make install}. For installation of \MBSim{} one types
\begin{verbatim}
 cd $HOME/MBSim
 svn checkout http://mbsim-env.googlecode.com/svn/trunk/ mbsim
 cd $HOME/MBSim/mbsim/kernel
\end{verbatim}
Invoke the procedures \textsc{automake, configure} for dynamic compilation with the prefix
\begin{verbatim}
--prefix=$HOME/MBSim/Install
\end{verbatim}
as well as FLAGs 
\begin{verbatim}
  --with-casadi-inc-prefix="path-to-casadi-includes"
  --with-casadi-lib-prefix="path-to-casadi-libs"
\end{verbatim}
\texttt{make doc} and \textsc{install} to install the basic module and to create a Doxygen HTML class documentation. In
\begin{verbatim}
$HOME/MBSim/mbsim/kernel/xmldoc
\end{verbatim}
invoke \textsc{install} for an XML documentation in \texttt{\$HOME/MBSim/Install/share/mbxmlutils/doc}.\vspace{5mm}
%---------MODULES------------
The following modules are available in MBSim:
\begin{itemize}
\item mbsimControl
\item mbsimHydraulics
\item mbsimFlexibleBody
\item mbsimElectronics
\item mbsimPowerTrain
\end{itemize}

The installation proceeds as follows:
\begin{verbatim}
 cd $HOME/MBSim/mbsim/modules/mbsimControl
\end{verbatim}
Invoke the procedures \textsc{automake, configure} for dynamic compilation with the prefix
\begin{verbatim}
--prefix=$HOME/MBSim/Install
\end{verbatim}
\texttt{make doc} and \textsc{install} to install the signal processing and control module and to create a Doxygen HTML class documentation. In 
\begin{verbatim}
$HOME/MBSim/mbsim/modules/mbsimControl/xmldoc
\end{verbatim}
invoke \textsc{install} for an XML documentation in \texttt{\$HOME/MBSim/Install/share/mbxmlutils/doc}.\vspace{5mm}
%---------------
\begin{verbatim}
 cd $HOME/MBSim/mbsim/modules/mbsimHydraulics
\end{verbatim}
Invoke the procedures \textsc{automake, configure} for dynamic compilation with the prefix
\begin{verbatim}
--prefix=$HOME/MBSim/Install
\end{verbatim}
\texttt{make doc} and \textsc{install} to install the hydraulics module and to create a Doxygen HTML class documentation. In 
\begin{verbatim}
$HOME/MBSim/mbsim/modules/mbsimHydraulics/xmldoc
\end{verbatim}
invoke \textsc{install} for an XML documentation in \texttt{\$HOME/MBSim/Install/share/mbxmlutils/doc}.\vspace{5mm}
%---------------
\begin{verbatim}
 cd $HOME/MBSim/mbsim/modules/mbsimFlexibleBody
\end{verbatim}
Invoke the procedures \textsc{automake, configure} for dynamic compilation with the prefix
\begin{verbatim}
--prefix=$HOME/MBSim/Install
\end{verbatim}
\texttt{make doc} and \textsc{install} to install the flexible body module and to create a Doxygen HTML class documentation. In 
\begin{verbatim}
$HOME/MBSim/mbsim/modules/mbsimFlexibleBody/xmldoc
\end{verbatim}
invoke \textsc{install} for an XML documentation in \texttt{\$HOME/MBSim/Install/share/mbxmlutils/doc}.\vspace{5mm}
%---------------
\begin{verbatim}
 cd $HOME/MBSim/mbsim/modules/mbsimElectronics
\end{verbatim}
Invoke the procedures \textsc{automake, configure} for dynamic compilation with the prefix
\begin{verbatim}
--prefix=$HOME/MBSim/Install
\end{verbatim}
\texttt{make doc} and \textsc{install} to install the electronics module and to create a Doxygen HTML class documentation.\vspace{5mm}
%---------------
\begin{verbatim}
 cd $HOME/MBSim/mbsim/modules/mbsimPowerTrain
\end{verbatim}
Invoke the procedures \textsc{automake, configure} for dynamic compilation with the prefix
\begin{verbatim}
--prefix=$HOME/MBSim/Install
\end{verbatim}
\texttt{make doc} and \textsc{install} to install the module for gears and to create a Doxygen HTML class documentation.\vspace{5mm} 

%---------------MBSimXML---------------
Finally mbsimxml has to be installed.
\begin{verbatim}
 cd $HOME/MBSim/mbsim/mbsimxml
\end{verbatim}
Invoke the procedures \textsc{automake, configure} for dynamic compilation with the prefix
\begin{verbatim}
--prefix=$HOME/MBSim/Install
\end{verbatim}
and \textsc{install} to install the XML module which contains an executable to invoke the preprocessor and after that evaluate the resulting flat structure. In
\begin{verbatim}
$HOME/MBSim/mbsim/mbsimxml/xmldoc
\end{verbatim}
invoke \textsc{install} for an XML documentation in \texttt{\$HOME/MBSim/Install/share/mbxmlutils/doc}.

\subsubsection{\MBSim Examples}
The examples are used for testing successful installation. There are two possibilities:
\begin{enumerate}
\item Change to the specific directory \texttt{\$HOME/MBSim/mbsim/examples/*} and type \texttt{make} to create an executable. The simulation starts with the command~\texttt{./main}. The results are visualised with the command~\texttt{openmbv} and plotted with~\texttt{h5plotserie} after having installed the visualisation framework (cf.~Sec.~\ref{sec:install:visualisation} and \ref{sec:plot}).
\item Use the script \texttt{./runexamples.sh install} in \texttt{\$HOME/MBSim/mbsim/examples} to install reference files. Then \texttt{./runexamples.sh} compiles, runs and tests each example. See \texttt{./runexamples.sh -h} for additional information.
\end{enumerate}

%%------------------------------------------------------------ SUBSECTION ---------------------
\subsection{Installation of the Visualisation Framework\label{sec:install:visualisation}}
In the following it is assumed, that a directory~\texttt{OpenMBV} and a directory \texttt{OpenMBV/Install} has been created in the \texttt{\$HOME} path of the Linux operating system.\par
%
Invoke 
\begin{verbatim}
 ssh otto (only at the institute)
 export PKG_CONFIG_PATH=/home/OpenMBV/local/lib/pkgconfig:
    $HOME/OpenMBV/Install/lib/pkgconfig (only at the institute)
\end{verbatim}
to change to a computer defining minimal requirements for further usage of the executable files and having installed the necessary software (cf. Sec.~\ref{sec:third_party}).

This subsection describes a \textbf{static} compilation, therefore the additional FLAG have to be used in each step
\begin{verbatim}
 --disable-shared --enable-static
\end{verbatim}

\subsubsection{\HDF}
Install \HDF{} and the \HDF{} wrapper as described in Sec.~\ref{sec:install:simulation} but in the directory \texttt{OpenMBV} and using a static compilation.\par
For convenient plotting of \HDF{} files it is assumed that Qwt with version 5 or newer is installed.\par
Invoke 
\begin{verbatim}
 cd $HOME/OpenMBV/HDF5Serie/h5plotserie
\end{verbatim}
as well as the procedures \textsc{automake, configure} for static compilation, \texttt{make doc} and \textsc{install} for installation and creation of a Doxygen HTML class documentation. Then, \texttt{.bashrc} can be extended with
\begin{verbatim}
alias h5plotserie="$HOME/OpenMBV/Install/bin/h5plotserie"
\end{verbatim}
to gain overall access to the command \texttt{h5plotserie}.

\subsubsection{\OpenMBV{}}
\paragraph{XML Utils}
Install XML Utils as described in Sec.~\ref{sec:install:simulation} but in the directory \texttt{OpenMBV} and using a static compilation.

\paragraph{OpenMBV-C++Interface}
Install OpenMBV-C++Interface as described in Sec.~\ref{sec:install:simulation} but in the directory \texttt{OpenMBV} and using a static compilation with the following additional FLAG
\begin{verbatim}
 --with-qwt-inc-prefix=/home/OpenMBV/local/include 
 --with-qwt-lib-prefix=/home/OpenMBV/local/lib
\end{verbatim}

\paragraph{\OpenMBV{}}
There is a static Linux binary available at \url{http://code.google.com} being updated from time to time. For the installation of a static visualisation using always the newest source files it is assumed that
\begin{itemize}
\item Coin3d with version 3 or newer 
\item hdf5 with version 1.8.2 or newer 
\item Qt with version 4.4 or newer 
\item HDF5Serie 
\item SoQt with version 1.4.1 or newer 
\item Qwt with version 5 or newer 
\end{itemize}
is installed.\par
With
\begin{verbatim}
 cd $HOME/OpenMBV/OpenMBV/openmbv
\end{verbatim} 
and the procedures \textsc{automake, configure} for static compilation with the additional FLAG
\begin{verbatim}
 --with-qwt-inc-prefix=/home/OpenMBV/local/include 
 --with-qwt-lib-prefix=/home/OpenMBV/local/lib CXXFLAGS="--Wno-non-virtual-dtor" 
\end{verbatim}
\texttt{make doc} and \textsc{install} a static build of the viewer together with an Doxygen HTML class documentation completes the installation. Then,
\begin{verbatim}
 exit (only at the institute)
\end{verbatim}
Last, \texttt{.bashrc} can be extended with
\begin{verbatim}
 alias openmbv="$HOME/OpenMBV/Install/bin/openmbv"
\end{verbatim}
to gain overall access to the command \texttt{openmbv}, which should be used only locally because of network protocols not providing the necessary X.org requirements.

